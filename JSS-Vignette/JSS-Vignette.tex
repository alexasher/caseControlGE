\documentclass[nojss]{jss}
\usepackage[utf8]{inputenc}

\providecommand{\tightlist}{%
  \setlength{\itemsep}{0pt}\setlength{\parskip}{0pt}}

\author{
Alex Asher\\Texas A\&M University
}
\title{Semiparametric Analysis of Polygenic Gene-Environment Interactions in
Case-Control Studies with \pkg{caseControlGE}}

\Plainauthor{Alex Asher}
\Plaintitle{Semiparametric Analysis of Polygenic Gene-Environment Interactions in
Case-Control Studies with caseControlGE}
\Shorttitle{\pkg{caseControlGE}: Gene-Environment Interactions in Case-Control
Studies}

\Abstract{
Standard logistic regression analysis of case-control data has low power
to detect gene-environment interactions, but until recently it was the
only method that could be used on complex polygenic data for which
parametric distributional models are not feasible. Under the assumption
of gene-environment independence in the underlying population, Stalder
et. al. (2017, \emph{Biometrika}, \textbf{104}, 801-812) developed a
retrospective method that treats both genetic and environmental
variables nonparametrically. We propose an improvement to the method of
Stalder et. al. that increases the efficiency of the estimates with no
additional assumptions and modest computational cost. This improvement
is achieved by treating the genetic and environmental variables
symmetrically to generate two sets of parameter estimates that are
combined to generate a more efficient estimate. We employ a a
semiparametric framework to develop the asymptotic theory of the
estimator, and evaluate its performance via simulation studies. The
method is illustrated using data from a case-control study of breast
cancer.
}

\Keywords{case-control study; gene-environment interaction; genetic epidemiology;
retrospective method; semiparametric analysis; pseudolikelihood;
polygenic analysis}
\Plainkeywords{case-control study; gene-environment interaction; genetic epidemiology;
retrospective method; semiparametric analysis; pseudolikelihood;
polygenic analysis}

%% publication information
%% \Volume{50}
%% \Issue{9}
%% \Month{June}
%% \Year{2012}
%% \Submitdate{}
%% \Acceptdate{2012-06-04}

\Address{
    Alex Asher\\
  Texas A\&M University\\
  Department of Statistics\\
  E-mail: \email{alexasher@stat.tamu.edu}\\
  
  }

\usepackage{amsmath}

\begin{document}

\begin{titlepage}
\end{titlepage}

\section{Introduction}\label{introduction}

\section{more stuff}

This template demonstrates some of the basic latex you'll need to know
to create a JSS article.

\subsection{Code formatting}\label{code-formatting}

Don't use markdown, instead use the more precise latex commands:

\begin{itemize}
\tightlist
\item
  \proglang{Java}
\item
  \pkg{plyr}
\item
  \code{print("abc")}
\end{itemize}

\section{R code}\label{r-code}

Can be inserted in regular R markdown blocks.

\begin{CodeChunk}

\begin{CodeInput}
R> x <- 1:10
R> x
\end{CodeInput}

\begin{CodeOutput}
 [1]  1  2  3  4  5  6  7  8  9 10
\end{CodeOutput}
\end{CodeChunk}

\section{trying more sections}\label{trying-more-sections}

sdfdsdsfdsdf

\subsection{subsection}\label{subsection}

sdfsdf

\begin{CodeChunk}

\begin{CodeInput}
R> cars
\end{CodeInput}

\begin{CodeOutput}
   speed dist
1      4    2
2      4   10
3      7    4
4      7   22
5      8   16
6      9   10
7     10   18
8     10   26
9     10   34
10    11   17
11    11   28
12    12   14
13    12   20
14    12   24
15    12   28
16    13   26
17    13   34
18    13   34
19    13   46
20    14   26
21    14   36
22    14   60
23    14   80
24    15   20
25    15   26
26    15   54
27    16   32
28    16   40
29    17   32
30    17   40
31    17   50
32    18   42
33    18   56
34    18   76
35    18   84
36    19   36
37    19   46
38    19   68
39    20   32
40    20   48
41    20   52
42    20   56
43    20   64
44    22   66
45    23   54
46    24   70
47    24   92
48    24   93
49    24  120
50    25   85
\end{CodeOutput}
\end{CodeChunk}



\end{document}

